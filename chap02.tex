% 论文第二章
\renewcommand{\baselinestretch}{1.5}
\section{PWM逆变器工作原理}
\subsection{二级标题}
\subsubbsection{2.1.1}{三级标题}
\zihao{-4}
正文中的公式。采用“equation”环境进行编辑。
\begin{equation}
E_i = \int|c_i(t)|^2 dt = \sum_{j=1}^{m}|c_{ij}|^2
\end{equation}
其中, $i =1,2,\dots,n,j=1,2,\dots,m$表示IMF的离散点的幅值,为信号采样点的数目。
(公式中涉及的字母所代表的物理量要全部指明。)\par
正文中的图,采用“figure”环境进行编辑。(图题在图的下方,居中,图题文字为五号宋体,图中文字为小五号)\par
\begin{figure}[htbp]
\centering
\includegraphics[width=9cm]{example-image-a.pdf}
\caption{功率波动的概率密度分布曲线图}     \label{fig1}
\end{figure}

正文中的表,采用“table”等环境进行编辑,建议采用Excel2lateX宏包直接在Excel中生成代码,设为1.2倍行距,标题和表格间距-5 pt。(采用三线表,表题为五号宋体,表中为小五宋体。表题在表上方,居中)
\begin{table}[htbp]
  \centering
  \caption{三种流型各模态能量特征表}
  \vspace{-5pt}
  \begin{spacing}{1.2}
    \begin{tabular}{ccccccccc}
    \hline
    流型    & IMF1  & IMF2  & IMF3  & IMF4  & IMF5  & IMF6  & IMF7  & IMF8 \bigstrut\\
    \hline
    A     & 0.5284 & 0.3418 & 0.055 & 0.0449 & 0.0175 & 0.0026 & 0.0039 & 0.0058 \bigstrut[t]\\
    A     & 0.5326 & 0.3406 & 0.0571 & 0.0423 & 0.0155 & 0.0024 & 0.0036 & 0.0059 \\
    B     & 0.458 & 0.308 & 0.0914 & 0.0623 & 0.0312 & 0.0323 & 0.0096 & 0.0082 \\
    B     & 0.4542 & 0.316 & 0.0919 & 0.0646 & 0.0345 & 0.0226 & 0.0068 & 0.0095 \\
    C     & 0.5782 & 0.2105 & 0.1041 & 0.0439 & 0.0476 & 0.0053 & 0.0077 & 0.0028 \\
    C     & 0.5762 & 0.2155 & 0.1006 & 0.0421 & 0.0502 & 0.0043 & 0.0087 & 0.0034 \bigstrut[b]\\
    \hline
    \end{tabular}%
  \end{spacing}
  \label{tab:addlabel}%
\end{table}%
