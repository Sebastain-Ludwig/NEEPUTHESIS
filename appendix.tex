\section*{附\quad 录}
\addcontentsline{toc}{section}{附\quad 录}
\renewcommand{\baselinestretch}{1.5}
\zihao{-4}
对于一些不宜放入正文中、但作为毕业设计(论文)又不可残缺,的组成部分,或有主要参考价值的内容,可编入毕设计(论文)的附录中,例如,公式的推演、编写的算法语言程序等。如果毕业设计中引用的实例、数据资料,实验结果等符号较多时,为了节约篇幅,便于读者查阅,可以编写一个符号说明,注明符号代表的意义。附录的篇幅不宜太多,附录一般不要超过正文。\par