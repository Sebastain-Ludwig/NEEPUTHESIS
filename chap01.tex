\neepuprolegomenon%输出绪论标题
\subsection{课题背景及研究的意义}%
\bodystyle
\subsubsection{课题背景}
50年代末晶闸管标志着电力电子半导体期间的开端。电力电子器件经历了40多年的发展历程\upcite{ref1,ref2},特别是近30多年内更是得到了迅猛的发展\upcite{ref3,ref4}。以Th(SCR)为代表的半控型器件是第一代电力电子器件\upcite{ref5},其主要用于可控整流装置,若用于可控的逆变器,因其无法自行关断,须配置强迫换流电路,致使装置复杂化。70年代中期,相继研制成功的电流控制型的双极性晶体管(Bipolar Junction Transistor——BJT)、门极可关断晶闸管(Gate Turn-off Thyristor——GTO)以及电压控制型的电力场效应晶体管(Power MOS Field-effect Transistor——P-MOSFET)等全控型器件被称为第二代电力电子器件\upcite{ref6}。\par

\subsubsection{课题研究的意义}
由于PWM逆变器的广泛应用及谐波会产生上述的诸多危害,因此必须对PWM逆变器的主电路及其谐波抑制技术进行研究。\par


\subsection{PWM逆变器研究现状}

所谓逆变器,是指整流(又称顺变)器的逆向变换装置。作为现代电力电子技术中最基本装置之一的PWM电压型逆变器是随着器件和控制技术的发展而不断发展起来的,采用PWM逆变技术的目的是为了获得不同或变化形式的电能。早期的半导体器件是普通的晶闸管半控型器件,其开关频率很低,逆变输出的交流电压的波形基本上是方波型。

\subsection{本文完成的主要工作}
综上所述,为了进一步提高应用最为广泛的SPWM电压型逆变器的性能,获得良好的经济和社会效益,必须对其主电路及谐波消除调制技术进行研究,解决SPWM逆变器主电路及谐波等问题。\par
本文所要完成的主要内容包括以下几个方面:

\begin{enumerate}[label=(\arabic*),topsep=0pt,itemsep=0pt,parsep=0pt,leftmargin=1.5cm]                                                                                                                                      
\item 传统SPWM电压型逆变器为减……                                                                                                                                            
\item 传统的SPWM…… 
\item 对本文所提……                                                                                                                                          
\end{enumerate}    