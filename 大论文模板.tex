%-*- coding:UTF*8 -*-
% 大论文.tex
%Copyright © 2022-2023 jycnb-666 and J.K Hao

%使用NEEPU_THSIS论文模板,参数与ctexart相同(通过NEEPU_THESIS将传递给ctexart)
\documentclass{NEEPU_THESIS}

%文章开始
\begin{document}
% 插入封面,参数从上到下分别是 文章标题第一行,标题第二行,学生姓名,班级,学号,指导就教师,所在单位,日期
%均无默认参数
%不填写的位置直接空出来
%必须按照顺序填写
%封面输出需要优化
\makecover{虚拟同步发电机}{VSG}{学生姓名}{电自卓越191}{2019301011016}{指导教师}{电气工程学院}{2023.2.5}

\newpage
\setcounter{page}{1}    % 重置页码
\pagenumbering{Roman}   % 设置正文页码格式

% 插入中文摘要
\neepupagestylea
\begin{Abstractcn}
    随着电力电子器件的发展,PWM电压型逆变器在交流变频调速、UPS、电能质量控制器、轻型直流输电换流器等电力电子装置中得到了越来越广泛的应用。PWM电压型逆变器直流侧所需的理想无脉动直流电压源通常通过整流加上大直流电容滤波获得。大直流滤波电容的使用,给装置带来占用空间大、成本高及严重影响电能质量方面的问题。因此,研究如何减小甚至去除逆变器直流侧电容,以及解决因其产生的低次谐波和相关问题,具有十分重要的理论意义和实用价值。本文在综述了国内外在PWM电压型逆变器及各种抑制谐波PWM技术的基础上,对目前工程中应用最广泛的SPWM电压型逆变器的主电路及谐波消除调制技术和相关问题进行了深入研究。
\end{Abstractcn}
%中文关键字
%这个版本中摘要和关键字之间的行距暂不满足要求
\begin{Keywordscn}
    PWM逆变器;直流电容;调制波重构SPWM;六脉动直流电压
\end{Keywordscn}

\newpage
% 插入英文摘要
\neepupagestylea
\begin{Abstracten}
    With the rapid development of the power and electron parts, PWM voltage inverter is applied, on a larger and larger scale, to ac frequency control system、UPS、power quality controller and Converter of highlight HVDC transmission. Non-ripple DC voltage source of PWM voltage type Inverter is gained by rectifiers paralleled connection a large DC filter capacitor. The equipment that uses a large DC filter capacitor has disadvantages, such as bulk, high price and low power quality. It is therefore becoming most significant and valuable to do research on how to reduce capacitance as small as possible or even remove capacitor and how to settle relative issues and to eliminate low order harmonics due to minishing capacitance or removing capacitor. Based on the survey of main circuits of inverter and restraining harmonic effects of PWM techniques, the main circuits and harmonic elimination PWM techniques of SPWM voltage type inverter, which applied widely in industry fields, are profoundly analyzed.
\end{Abstracten}
%插入英文关键词
\begin{Keywordsen}
    {PWM inverter; DC capacitor; Six-ripple DC voltage; Modulation wave reconstruction SPWM}
\end{Keywordsen}
\newpage

%输出目录
\addcontentsline{toc}{section}{目\quad 录}
\begin{spacing}{1.5}
\tableofcontents\thispagestyle{fancy}       % 避免目录第二页没有页眉页脚
\end{spacing}



%% 正文从此开始
\newpage

%正文的页面格式设置
\neepupagestylechap

% 第一章
\neepuprolegomenon%输出绪论标题
\subsection{课题背景及研究的意义}%
\bodystyle
\subsubsection{课题背景}
50年代末晶闸管标志着电力电子半导体期间的开端。电力电子器件经历了40多年的发展历程\upcite{ref1,ref2},特别是近30多年内更是得到了迅猛的发展\upcite{ref3,ref4}。以Th(SCR)为代表的半控型器件是第一代电力电子器件\upcite{ref5},其主要用于可控整流装置,若用于可控的逆变器,因其无法自行关断,须配置强迫换流电路,致使装置复杂化。70年代中期,相继研制成功的电流控制型的双极性晶体管(Bipolar Junction Transistor——BJT)、门极可关断晶闸管(Gate Turn-off Thyristor——GTO)以及电压控制型的电力场效应晶体管(Power MOS Field-effect Transistor——P-MOSFET)等全控型器件被称为第二代电力电子器件\upcite{ref6}。\par

\subsubsection{课题研究的意义}
由于PWM逆变器的广泛应用及谐波会产生上述的诸多危害,因此必须对PWM逆变器的主电路及其谐波抑制技术进行研究。\par


\subsection{PWM逆变器研究现状}

所谓逆变器,是指整流(又称顺变)器的逆向变换装置。作为现代电力电子技术中最基本装置之一的PWM电压型逆变器是随着器件和控制技术的发展而不断发展起来的,采用PWM逆变技术的目的是为了获得不同或变化形式的电能。早期的半导体器件是普通的晶闸管半控型器件,其开关频率很低,逆变输出的交流电压的波形基本上是方波型。

\subsection{本文完成的主要工作}
综上所述,为了进一步提高应用最为广泛的SPWM电压型逆变器的性能,获得良好的经济和社会效益,必须对其主电路及谐波消除调制技术进行研究,解决SPWM逆变器主电路及谐波等问题。\par
本文所要完成的主要内容包括以下几个方面:

\begin{enumerate}[ref1label=(\arabic*),topsep=0pt,itemsep=0pt,parsep=0pt,leftmargin=1.5cm]                                                                                                                                      
\item 传统SPWM电压型逆变器为减……                                                                                                                                            
\item 传统的SPWM…… 
\item 对本文所提……                                                                                                                                          
\end{enumerate}    

% 第二章
% 论文第二章
\bodystyle
\section{PWM逆变器工作原理}
\subsection{二级标题}
\subsubsection{三级标题}
正文中的公式。采用“equation”环境进行编辑。
\begin{equation}
E_i = \int|c_i(t)|^2 dt = \sum_{j=1}^{m}|c_{ij}|^2
\end{equation}
其中, $i =1,2,\dots,n,j=1,2,\dots,m$表示IMF的离散点的幅值,为信号采样点的数目。
(公式中涉及的字母所代表的物理量要全部指明。)\par
正文中的图,采用“figure”环境进行编辑。(图题在图的下方,居中,图题文字为五号宋体,图中文字为小五号)\par
\begin{figure}[htbp]
\centering
\includegraphics[width=9cm]{example-image-a.pdf}
\caption{功率波动的概率密度分布曲线图}     \label{fig1}
\end{figure}

正文中的表,采用“table”等环境进行编辑,建议采用Excel2lateX宏包直接在Excel中生成代码,设为1.2倍行距,标题和表格间距-5 pt。(采用三线表,表题为五号宋体,表中为小五宋体。表题在表上方,居中)
\begin{NEEPUtritable}{cccccccccc}{三种流型各模态能量特征表}
    \hline
    流型    & IMF1  & IMF2  & IMF3  & IMF4  & IMF5  & IMF6  & IMF7  & IMF8 \bigstrut\\
    \hline
    A     & 0.5284 & 0.3418 & 0.055 & 0.0449 & 0.0175 & 0.0026 & 0.0039 & 0.0058 \bigstrut[t]\\
    A     & 0.5326 & 0.3406 & 0.0571 & 0.0423 & 0.0155 & 0.0024 & 0.0036 & 0.0059 \\
    B     & 0.458 & 0.308 & 0.0914 & 0.0623 & 0.0312 & 0.0323 & 0.0096 & 0.0082 \\
    B     & 0.4542 & 0.316 & 0.0919 & 0.0646 & 0.0345 & 0.0226 & 0.0068 & 0.0095 \\
    C     & 0.5782 & 0.2105 & 0.1041 & 0.0439 & 0.0476 & 0.0053 & 0.0077 & 0.0028 \\
    C     & 0.5762 & 0.2155 & 0.1006 & 0.0421 & 0.0502 & 0.0043 & 0.0087 & 0.0034 \bigstrut[b]\\
    \hline
\end{NEEPUtritable}

%更多章节的书写参考以上两个章节

% 结论
\addcontentsline{toc}{section}{结\quad 论}
\section*{结\quad 论}
\begin{spacing}{1.5}

\zihao{-4}
PWM逆变器的应用已遍及各个工业领域,其整流侧大直流电容所带来的负面效应日益受到重视。研究如何尽量减小甚至去掉它以及由此所带来的谐波和相关问题及其解决方法具有十分重要的理论意义和实用价值。本文在综述了国内外对PWM逆变器及其谐波消除研究现状的基础上,对目前工程中应用最广泛的SPWM电压型逆变器的主电路及其谐波消除调制技术等基本问题进行了深入研究。\par
论文的主要工作如下。
\begin{itemize}[topsep=0pt,itemsep=0pt,parsep=0pt,leftmargin=1.5cm]
\item[(1)] 提出了……。
\item[(2)] 对……。
\item[(3)] 运用数学傅立叶级数理论。
\end{itemize}

\end{spacing}

% 参考文献
\addcontentsline{toc}{section}{参考文献}
\begin{spacing}{1.5}
\zihao{-4}
\begin{thebibliography}{99}

\bibitem{ref1} 娄素华,卢斯煜,吴耀武,尹项根.低碳电力系统规划与运行优化研究综述[J].电网技术,2013,37(6):1483-1490.
\bibitem{ref2} 王鸣晓,林建民,马光进,周淑玲,董涛.电气设备防雷工程设计探讨[J].气象科技,2013,41(2):417-421.
\bibitem{ref3} 赵洪涛.电力系统规划分类及方法探究[J].中国电力教育,2013,11:184-185.
\bibitem{ref4} 关蕾.绥中电厂2×1000MW超超临界机组工程造价估算方法研究[D].华北电力大学(北京),2010.
\bibitem{ref5} 张慧敏.试论天津地区燃煤电厂的大气污染及其治理对策[J].天津电力技术,2000,3:19-21. 
\bibitem{ref6} 周小谦.我国“西电东送”的发展历史、规划和实施[J].电网技术,2003,27(5):1-5.
\bibitem{ref7} 陈跃.电力工程专业毕业设计指南(电力系统分册) [M].北京:中国水利水电出版社,2003.
\bibitem{ref8} Lin Weixing,Wen Jinyu,Hu Jianglu,et al.An investigation on the active-power variations of wind farms[J].IEEE Transactions on Industry Applications,2012,48(3),1087-1094.

\end{thebibliography}

\end{spacing}

% 致谢
\section*{致\quad 谢}
\addcontentsline{toc}{section}{致\quad 谢}
\renewcommand{\baselinestretch}{1.5}
\zihao{-4}
致谢应以简短的文字对课题研究与论文撰写过程中曾直接给予帮助的人员(例如指导教师、答疑教师及其他人员)表示自已的谢意,这不仅是一种礼貌,也是对他人劳动的尊重,是治学者应有的思想作风。\par

% 附录
\neepuappendix%输出附录
\bodystyle%附录的格式规定
对于一些不宜放入正文中、但作为毕业设计(论文)又不可残缺,的组成部分,或有主要参考价值的内容,可编入毕设计(论文)的附录中,例如,公式的推演、编写的算法语言程序等。如果毕业设计中引用的实例、数据资料,实验结果等符号较多时,为了节约篇幅,便于读者查阅,可以编写一个符号说明,注明符号代表的意义。附录的篇幅不宜太多,附录一般不要超过正文。\par

\end{document}